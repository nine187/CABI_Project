
\documentclass[a4paper,11]{article}

%\title{Unused Title}
\usepackage{graphicx}
\usepackage{hyperref}
\usepackage{multirow}
\usepackage{multicol}
\usepackage{blindtext}
\usepackage[utf8]{inputenc}
\usepackage[english]{babel}
\usepackage[T1]{fontenc}
\usepackage{geometry}
% Use helvet if uarial cannot be installed
%\usepackage{uarial}
\usepackage[scaled]{helvet}

\renewcommand{\familydefault}{\sfdefault}
\usepackage{amssymb}
\usepackage{amsmath}
\usepackage{courier}
\usepackage{setspace}
\usepackage[table,svgnames]{xcolor}
\usepackage{fancyvrb} 
\usepackage{listings}
\usepackage{caption}
\usepackage{longtable}
\usepackage{relsize}
\usepackage{tfrupee}
\usepackage{rotating}
\usepackage{lipsum}
\usepackage{subcaption}
\usepackage{float}
\usepackage{aliascnt}
\usepackage{amsmath}

\usepackage{natbib}
\newcommand*{\urlprefix}{Available from: }
\newcommand*{\urldateprefix}{Accessed }
\bibliographystyle{agsm}
\makeatletter
\newcommand\footnoteref[1]{\protected@xdef\@thefnmark{\ref{#1}}\@footnotemark}
\makeatother

\newaliascnt{eqfloat}{equation}
\newfloat{eqfloat}{h}{eqflts}
\floatname{eqfloat}{Equation}

\newcommand*{\ORGeqfloat}{}
\let\ORGeqfloat\eqfloat
\def\eqfloat{%
	\let\ORIGINALcaption\caption
	\def\caption{%
		\addtocounter{equation}{-1}%
		\ORIGINALcaption
	}%
	\ORGeqfloat
}

\addto\captionsenglish{% Replace "english" with the language you use
	\renewcommand{\contentsname}%
	{List of Contents}%
}

\newcommand\tab[1][1cm]{\hspace*{#1}}

\definecolor{codegreen}{rgb}{0,0.6,0}
\definecolor{codegray}{rgb}{0.5,0.5,0.5}
\definecolor{codepurple}{rgb}{0.58,0,0.82}
\definecolor{backcolour}{rgb}{0.95,0.95,0.92}

\lstdefinestyle{mystyle}{
	backgroundcolor=\color{backcolour},   
	commentstyle=\color{codegreen},
	keywordstyle=\color{magenta},
	numberstyle=\tiny\color{codegray},
	stringstyle=\color{codepurple},
	basicstyle=\ttfamily\footnotesize,
	breakatwhitespace=false,         
	breaklines=true,                 
	captionpos=b,                    
	keepspaces=true,                 
	numbers=left,                    
	numbersep=5pt,                  
	showspaces=false,                
	showstringspaces=false,
	showtabs=false,                  
	tabsize=2,
	xleftmargin=2.0 cm,
	xrightmargin=-2.0cm,
	frame=lr,
	%	framesep=-5pt,
	framerule=0pt
}

\lstset{style=mystyle}

\definecolor{Teal}{RGB}{0,128,128}
\definecolor{NewBlue1}{RGB}{4,100,226}
\definecolor{NiceBlue}{RGB}{63,104,132}
\definecolor{DarkRed}{RGB}{14,53,59}
\definecolor{NewBlue2}{RGB}{62,100,125}
\definecolor{NewBlue3}{RGB}{44,100,128}

\hypersetup{
	colorlinks,
	citecolor=NiceBlue,
	linkcolor=NewBlue1,
	urlcolor=Blue
	%	citebordercolor=Violet,
	%	filebordercolor=Red,
	%	linkbordercolor=Blue
}


% Modify the \citepp command to produce (Author, Year) format
\let\oldcitep\citepp
\renewcommand{\citepp}[1]{(\citepalp{,#1})}


\usepackage{geometry}
\linespread{1.25}
\usepackage[parfill]{parskip} % Avoid indentation

\geometry{
	a4paper,
	left=4cm,
	right=2.5cm,
	top=2.5cm,
	bottom=2.5cm,
}

\usepackage{fancyhdr}
\pagestyle{fancy}
\fancyhf{} % Clear all header and footer fields
\fancyhead[R]{} % Set the header content on the right side
\fancyfoot[C]{\thepage} % Set the page number at the center of the footer
\renewcommand{\headrulewidth}{0pt} % Remove header rule line

\begin{document}
	\pagenumbering{gobble}
	\begin{center}
		{\large IMPERIAL COLLEGE LONDON}
	\end{center}
	%	\maketitle
	\vspace{4cm}
	
	\begin{center}
		
		\Huge Modeling the spatial distribution of aflatoxin in maize crops and \textit{Aspergillus flavus} with CLIMEX and ensemble species distribution model\\	
  
		\vspace{.5cm}		

  
	\end{center}
	\vspace{2.5cm}
	\begin{center}
		\Large Pasith Prayoonrat\\External Supervisor: Dr. Anna Szyniszewska
        \\Internal Supervisor: Dr. Martin Bidartondo
	\end{center}
	
	\vspace{4cm}
	\begin{center}
		{\large A thesis submitted in partial fulfilment of the requirements for the degree of \\
        \\ Master of Science at Imperial College London Submitted for the \\ MSc Computational Methods in Ecology and Evolution}
	\end{center}
	
	\begin{center}
		{\large 24 August 2023}
	\end{center}		

\newpage
	\pagenumbering{Roman}

\section*{Declaration and Acknowledgement}
I declare the work presented in this thesis to be my own.
The code for this project can be accessed at \url{https://github.com/nine187/CABI_Project}.

\\
I want to thank my supervisor, Dr. Anna Szyniszewska, for all the guidance and resources provided throughout my time working on this project.  Also, I thank my friends who supported me emotionally while working on this project.

I sincerely thank Prof Darren Kriticos for his help consulting with the parameters in CLIMEX.

\\
The meteorological data were provided by Dr. Anna Szyniszewska of CABI. Additionally, parts of the code and R packages used for the analysis were provided and adapted from her.
\\
Bioclim variables were downloaded from the CliMond Archive \citeppp{al2012prevalence}, and the following variables were chosen: Bio1-Bio40 \citep{hutchinson2009anuclim} \citep{kriticos2014extending}.

\\
The word count of this thesis is 5610.



\section*{Lexicon}
\textbf{ANN}- Artificial neural network \\
\textbf{AUC}- Area under the ROC curve \\
\textbf{CABI}- Centre for Agricultural and Bioscience \\
\textbf{CLIMEX}- CLIMatic indEX \\
\textbf{EI}- Ecoclimatic index \\
\textbf{GAM}- Generalized addition model \\
\textbf{GBIF}- Global Biodiversity Information Facility \\
\textbf{GLM}- Generalized linear model \\
\textbf{MARS}- Mulitvariate adaptive regression spline \\
\textbf{MAXENT}- Maximum entropy \\
\textbf{SDM}- Species distribution model \\
\textbf{SVM}- Support vector machine \\
\textbf{RF}- Random forest


\section*{Keywords}
aflatoxin, \textit{Aspergillus flavus}, CLIMEX, ensemble model, climate change, species distribution \\ modeling

\pagebreak

\centering

\section*{ABSTRACT}
\raggedright % Stop centering here

\textit{Aspergillus flavus} is a species of fungus that produces a potent hepatocarcinogenic mycotoxin called aflatoxin. Aflatoxin can contaminate crops and is produced primarily in maize by the fungus species. In order to understand the distribution of the pathogen system in maize and \textit{Aspergillus flavus}, I used a mechanistic modeling tool: CLIMEX, and an ensemble species distribution model to determine its distribution range. I created a parameter involving specific temperature and moisture ranges where aflatoxin can be produced in maize by \textit{A. flavus}, then ran the parameters with historical and future meteorological data using a mechanistic species distribution computational tool: CLIMEX. To explore the species distribution of \textit{A. flavus}, I ran a pseudo-absence ensemble species distribution model comprised of seven different algorithms to understand the global distribution of \textit{A. flavus} with the Global Biodiversity Information Facility occurrence database. The models showed that the species were mainly prevalent in tropical and subtropical regions and will distribute in higher latitudes in the future due to a reduced cold stress level from climate change. I analyzed the changes in suitability of the historical data with high maize yields, and the trend showed increasing suitability of \textit{A. flavus} across areas with high maize yield. As the ecological niches of \textit{A. flavus} and maize changes in the future, areas in northern Europe and eastern Russia will be suitable to grow maize and susceptible to aflatoxin infection in the year 2100. Areas where aflatoxin is a threat will shift due to climate change, and modeling its main producing species \textit{A. flavus} is essential to reduce these highly carcinogenic metabolites' economic and health impacts.


\pagebreak
\tableofcontents
\pagebreak
\listoffigures
\listoftables
\pagebreak
\newpage
%\section{Notation}
%\textbf{IF APPLICABLE}

\cleardoublepage\pagenumbering{arabic}

\section{Introduction}
Aflatoxins is a group of secondary metabolites mycotoxins commonly found in tropical and subtropical regions \citepp{wu2015global}. Amongst all the aflatoxins, aflatoxin $B_{1}$ ($AFB_{1}$) is considered the most dangerous due to its hepatocarcinogenic property, \citep{gizachew2019aflatoxin} with the International Agency for Research on Cancer (IARC) classifying it as a group 1 carcinogen \citep{international2012review}. These mycotoxins are mainly produced by two species of fungus: \textit{Aspergillus flavus} producing aflatoxin $B_{1}$ and $B_{2}$ and \textit{Aspergillus parasiticus} producing aflatoxin $G_{1}$, $G_{2}$, and $M_{1}$ \citep{10.1079/cabicompendium.7432}. Aflatoxin was first discovered in the United Kingdom in 1961 after investigating mass turkey death caused by aflatoxin-contaminated groundnut feed \citep{aflatoxins1979other}. The mycotoxin is an agricultural and economic \citep{wu2015global} concern as it can contaminate a wide range of host plants, including various staple crop species \citep{10.1079/cabicompendium.7432}. For this reason, many studies \citep{chauhan2015improved} \citep{bernaldez2017influence}  had been conducted to investigate this toxin.

\begin{figure*}[!ht]
    \centering
    \begin{minipage}{10cm}
        \centering
        \includegraphics[width=\linewidth]{images/infected_corn.jpg}
        \caption{\textit{A. flavus} infected maize \\ Image Source: Denis C. McGee/Iowa State University \citep{10.1079/cabicompendium.7432} [last accessed 18/08/2023]}
        \label{fig:aflavus-infected-maize}
    \end{minipage}
\end{figure*}

One of the main producers of aflatoxin, the fungus species: \textit{A. flavus} is an airbone saprophyte fungus \citep{goldblatt2012aflatoxin} widely spread throughout the world in various forms \citep{diener1987epidemiology}. \textit{A. flavus} consists of 2 main morphological strains group: the S strain and the L strain, distinguishable by their sclerotia size \citep{amaike2011aspergillus}; due to the morpohological difference, both strains have slightly different ecological niches and aflatoxin production levels, resulting in the different toxigenicity level geographically \citep{10.1079/cabicompendium.7432}. The fungus can infect various crops during the pre or postharvest stage and produce a bright greenish-yellow fluorescent (BGYF) color (Figure 1), indicating the presence of the fungus; however, this does not prove that there is the presence of aflatoxin \citep{yao2006hyperspectral}. \vspace{\baselineskip} 

In countries that exports high amount of staple food crops, maize and peanuts are the two of the most contaminated from the aflatoxin \citep{scussel2006study}. The contaminations are usually dominated by two species of fungus \textit{A. parasiticus} and \textit{A. flavus} amongst different species of crops \citep{kumar2017aflatoxins}.  Thus for economic and health concerns, there has been much focus on the aflatoxin and the optimal conditions which could stimulate the growth of the fungus in pre and postharvest agricultural host plants \citep{chauhan2015improved} \citep{gizachew2019aflatoxin}  \citep{liu2021physico}. Moreover, due to the nature of the aflatoxin production being highly related to temperature \citep{arrus2005aflatoxin}, there has been extensive efforts \citep{haerani2020climate} \citep{valencia2020environment} in exploring a shift of the production range of these secondary carcinogenic metabolites globally due to climate change \citep{haerani2020climate}. The global temperature rise will change the species ecological niche, and the increased level of carbon dioxide from climate change has been associated with an increased level of aflatoxin production in maize \citep{gilbert2017carbon}. 
\vspace{\baselineskip} \\
With the shifting range of aflatoxin in the future, a better understanding of one of the main aflatoxin-producing species' \textit{A. flavus} distributions and their distribution in various crops are required. In order to do so, the use of different approaches of modeling to identify and verify areas where the toxin threatens crops is crucial. Therefore, I attempted to model the global range of the pathogen aflatoxin produced from \textit{A. flavus} in one of the major crops that are affected by aflatoxin contamination: maize and the global range of the fungus \textit{A. flavus }. In order to predict the global range of the aflatoxin in maize from the fungus, I used the mechanistic modeling tool: CLIMEX to predict the distribution of the species as the model uses the ecological niche of a species to predict the distribution. For the global distribution of \textit{A. flavus}, I used an ensemble Species Distribution modeling (SDM) technique comprising different algorithms with varying reliability \citep{valavi2022predictive} to determine the global range of \textit{A. flavus}. 
\vspace{\baselineskip} \\ 
Alongside the models, I used the occurrence and environmental data available in the literature and online databases (i.e., GBIF, CliMond) to validify and predict the distribution. I also explored the trends of the changes in the ecological niche of the fungus in high maize yield areas over the years and explored potential areas that involved significant changes in suitability. Apart from modeling the distribution with historical meteorological data, I applied the parameters to future climate models to explore areas where the potential change in the trend of the ecological niche of \textit{A. flavus} could occur.

\newpage
\section{Methods}
\subsection{Climate data}
For the historical meteorological data, I used CliMond historical dataset to model the suitable ecological niche for \textit{A. flavus}. CliMond climate dataset (CM30\_1995H\_V2) is a 30-year average climatology centered on 1995 \citep{kriticos2012climond}. Additionally, to investigate the changes in ecological niches of \textit{A. flavus} over the years, I used a set of annual meteorological, CRU TS4 weekly data for 1970-2019 with 30' spatial resolution \citep{mitchell2005improved}. 

\subsection{Soil and irrigation data}
To account for the irrigation scenarios in CLIMEX, I ran the analysis by modifying the amount of rainfall top-up in the model to 2.5 mm per day for 12 months. Additionally, I did not apply irrigation to a specific location if the location contained rainfall equal to or more than 2.5 mm day\textsuperscript{-1}. I used the Global Map of Irrigation Areas (GMIA) version 5.0 to identify significant irrigated areas under irrigation (greater than 10 ha). Then, I resampled the data to match the spatial resolutions of the CRU TS4 spatial resolution dataset. For the 30' cell, if the calculated irrigation was higher than 10 ha, the scenarios were used to create a composite map \citep{siebert2013update}.
\vspace{\baselineskip} 


\subsection{Occurrence data}
\begin{figure*}[!ht]
	\centering
	\includegraphics[width=\textwidth]{images/maize_map.png}
	\caption{\textit{A. flavus} occurrence map obtained from CABI database and Google Scholar}
	\label{fig:testimage2}
\end{figure*}

\begin{figure*}[!ht]
	\centering
	\includegraphics[width=\textwidth]{images/aflavus_map.png}
	\caption{\textit{A. flavus} occurrence map obtained from GBIF}
	\label{fig:testimage1}
\end{figure*}



To identify the presence of aflatoxin produced from \textit{A. flavus} in maize, I gathered the  occurrence data available from the Centre for Agricultural and Bioscience (CABI) database \citep{10.1079/cabicompendium.7432}. Then, I extracted the data from CABI database along with data gathered from an additional literature search for the presence of \textit{A. flavus} in maize using Google Scholar. I used the compiled literature data to map approximate locations of the presence of \textit{A. flavus} in maize (Figure 2). Finally, I exported and mapped the datasets in CLIMEX to validate the parameters of the species.
\pagebreak \\
Furthermore, to identify the presence of \textit{A. flavus}, I used the Global Biodiversity Information Facility (GBIF) database to gather the  occurrence data of \textit{A. flavus} (Figure 3). I imported the data points using the rgbif package Version 3.7.7 \citep{rgbif} in RStudio and mapped the presence-only data points using the tmap package Version 3.3-3 \citep{tmap}. For data cleaning of the GBIF database, I used the CoordinateCleaner package version 2.0-20 \citep{coordclean} in RStudio to remove erroneous data points. The package checked for data points with no coordinate, duplicated, poorly geo-referenced data points, and data points near GBIF headquarters. In addition, I exported the data points as a .KML file and visualized the data points on GoogleMap to identify further individual potentially erroneous data points. I used this data to identify the global presence locations of \textit{A. flavus} and performed an ensemble SDM.
\subsection{Modeling approaches}
To understand the species distribution range of \textit{A. flavus}, I performed a literature review of the ecological niche of \textit{A. flavus}. As the distribution of \textit{A. flavus} is highly affected by temperature \citep{arrus2005aflatoxin} and moisture  \citep{holmquist1983influence} \citep{diener1987epidemiology} \citep{jaime2010crop}  using methods that were suitable to predict climatic suitability such as CLIMEX was ideal for predicting the distribution of these species. Additionally, to explore the effects of different ecological variables on \textit{A. flavus}, I performed an ensemble SDM using GBIF  occurrence dataset and 40 CliMond bioclimatic variables \citep{kriticos2012climond}. Ensemble model has been shown to have a high predictive performance and was one of the most accurate SDM amongst various algorithms used for SDM \citep{valavi2022predictive}. 

\subsubsection{CLIMEX}


In order to define the parameter of the species, I gathered the ecological and biological literature of \textit{A. flavus} and maize to understand its climatic preference. I determined the species optimal temperature, marginal temperature, and temperature, which can cause different ecological stresses. Then, I used the data to construct a parameter defining its temperature and moisture range and indicated the accumulated stress level of the species in maize. CLIMEX calculates the stress level by computing four different stresses: cold stress, dry stress, heat stress, and wet stress into a single Annual Stress Index (SI) (Equation 1) to determine areas where the species is not suitable to live. The SI value alongside the annual growth index \text{GI}\textsubscript{A} (Equation 2) was used to compute a single value of Ecoclimatic Index (EI), which determined the suitability of the area for that given species (Equation 4). EI values ranged from 0-100, where 0-10 indicates marginal suitability, 10-30 indicates that the species can survive, and 30-100 indicates that the species can thrive within the climatic area \citep{byeon2018review}.

\begin{equation}
SI = \frac{{(1 - CS)}}{{100}} \cdot \frac{{(1 - DS)}}{{100}} \cdot \frac{{(1 - HS)}}{{100}} \cdot \frac{{(1 - WS)}}{{100}}
\end{equation}
\\
SI is the annual stress index while CS, DS, HS, WS are the annual cold, dry, heat, wet stress indices respectively.

\begin{equation}
    \text{GI}_{A} = 100 \sum_{i=1}^{52} \frac{TGI_{W_i}}{52}
\end{equation}

 \text{GI}_{A} \text{is the annual growth index}  \
 \text{TGI}_{W_i} \text{is the total weekly growth index}

\begin{equation}
    EI = \text{TGI}_{A} * SI
\end{equation}
EI is the ecoclimatic index while $\text{TGI}_{A}$  \textnormal{is the annual growth index } \textnormal{and SI is the annual stress index}

\vspace{\baselineskip} \\

I used the "Compare Locations (1 species)" function with the Composite.dx file in CLIMEX to determine the range of \textit{A. flavus} in maize. The "Compare Locations (1 Species)" function uses a monthly extended meteorological database of average variables from different locations. Policymakers often use the function to identify areas susceptible to invasive species or pests \citep{kriticos2015climex}. The function was also helpful in identifying where specific areas of studies could be further conducted (e.g., heat tolerance, optimal moisture level) to understand more about a species \citep{kriticos2015climex}.
\vspace{\baselineskip} \\

\begin{table}[!ht]
    \caption{CLIMEX Parameters for \textit{A. flavus} in maize} % Caption with numbering
    \centering
    \begin{tabular}{lll}
        \hline
        ~ & ~ & ~  \\
        \textbf{Index} & \textbf{Parameters} & \textbf{Values} \\ 
        ~ & ~ & ~  \\ \hline\hline
        ~ & ~ & ~  \\ 
        Temperature & DV0 & 10°C \\ 
        ~ & DV1 & 30°C \\ 
        ~ & DV2 & 35°C \\ 
        ~ & DV3 & 40°C \\ 
        Moisture & SM0 & 0.1 \\ 
        ~ & SM1 & 0.2 \\ 
        ~ & SM2 & 0.8 \\ 
        ~ & SM3 & 2 \\ 
        Cold stress & TTCS & 10°C \\ 
        ~ & THCS &  -0.00015 week\textsuperscript{-1} \\ 
        Heat stress & TTHS & 40°C \\ 
        ~ & THHS & 0.00004 week\textsuperscript{-1} \\ 
        Dry stress & SMDS & 0.1 \\ 
        ~ & HDS &  -0.0002 week\textsuperscript{-1} \\ 
        Wet stress & SMWS & 2 \\ 
        ~ & HWS & 0.0009 week\textsuperscript{-1} \\ 
        ~ & ~ & ~  \\ \hline
    \end{tabular}
\end{table}

\pagebreak
To fine-tune the parameters for the species, I used the historical meteorological dataset centered in 1995 alongside the  occurrence. I set the parameters (Table 1) for the pathogen model in this study based on (1) the range temperature and moisture where \textit{A. flavus} are found to produce aflatoxin from academic literature, (2) CABI's distribution map of \textit{A. flavus}, and (3) the advice from my supervisor and experts regarding the parameters. I started by defining the parameters based on previous academic literature, which utilizes CLIMEX in the viable agricultural areas of maize \citep{ramirez2017global} and aflatoxin presence in other crops \citep{haerani2020climate}. Then, I adjusted the temperature and moisture range using the basis range of aflatoxin production from the fungus I found in the literature. For the stress parameters, I adjusted the parameters to incorporate the unsuitable areas from temperature factors by iteratively fitting the parameters with the  occurrence data. I finalized the parameters (Table 1) and used the same parameters to model the species in the Composite Map of 1970-2019 and the future climate scenarios. I ran the parameter sensitivity (Supplementary material 2) function in CLIMEX to determine the effects of each variables on my output. The description of the parameters is as below. 

\textbf{Temperature index} \\ 
Chauhan et al.,(2015) modeled the aflatoxin risk in maize and found the marginal temperature and production range in maize to be 11.5°C to 42.5°C. The minimal values slightly differed from the study by Liu et al., (2021) which was at 10°C. I used these values as a baseline of the limiting temperature range (DV0 and DV3) and iteratively fit the parameter to acco mmodate the  occurrence data. I set the value of DV0 and DV3 at 10°C  and 40°C, respectively. I also cross-referenced the locations with the CABI database of occurrence to identify areas of \textit{A. flavus} that did not fit within the parameters. For the optimal temperature range (DV1 and DV2), \textit{A. flavus} grew at a higher temperature range compared to its optimum aflatoxin production range \citep{bernaldez2017influence}. The optimum aflatoxin production range was around 25-35°C depending on the water activity level  \citep{chauhan2015improved} \citep{mannaa2017influence} and the specific aflatoxin types (i.e. AFB and AFG) \citep{kumar2021aflatoxin}. I set the values of the optimal temperature (DV1 and DV2) in this study at 30-35°C, which was the range aflatoxin can be produced at 0.95$ a_w $ in Mannaa & Kim (2017) studies and its optimal growth temperature \citep{chuaysrinule2020comparative} \citep{norlia2020modelling}.  
\vspace{\baselineskip} \\
The temperature range for which \textit{A. flavus} can optimally grow and produce aflatoxin in different crops was different \citep{norlia2019aspergillus} due to various factors such as availability of the substrate \citep{gilbert2017carbon}, water activity, and temperature \citep{gizachew2019aflatoxin}. Aflatoxins were produced optimally at varying temperature ranges, and moisture \citep{gizachew2019aflatoxin}. I used the 30-35°C range from \citep{mannaa2017influence} studies in CLIMEX's ecological parameter as it is the temperature range where \textit{A. flavus} could occur at all the occurrence data points on the historical dataset and within the temperature range of the aflatoxin risk model in maize \citep{chauhan2015improved}.  
\vspace{\baselineskip} \\
\textbf{Moisture index} \\
CLIMEX uses the criteria to measure moisture by utilizing the percent of soil moisture. For instance, 10 percent of soil moisture would translate to 0.1. For the limiting low moisture (SM0) of the CLIMEX parameters, I set the value at 0.1 moisture level to account for the crop's wilting point as Haerani et al., (2020) CLIMEX's parameter of aflatoxin in peanut. For the optimal lower moisture (SM1) \citep{payne1983effect} suggested that the moisture content of maize grain before maturity is always more than 24 percent. Therefore, I rounded the moisture to 0.2 (personal co mmunication with CLIMEX modeling expert, Darren Kriticos) to use the moisture index with one decimal point and used this as a starting point, then iteratively fitted the data to \textit{A. flavus}  occurrence data in maize. I set the optimal higher moisture (SM2) to be 0.8 as Liu et al., (2020) suggested that the suitable moisture for aflatoxin is around 0.8-0.9. For the limiting high moisture (SM2), I set the value at 2 to accumulate for the Wet Stress, same as Haerani et al., (2020) model.
\vspace{\baselineskip} \\
\textbf{Cold stress}
\\
To determine areas where cold stress limited the distribution of aflatoxin in maize, I set the cold stress degree day threshold (DTCS) at 10°C, the same value as DV0. Then, I iteratively modified DHCS to -0.00015 week\textsuperscript{-1} to include the area in the southern part of Europe (i.e., Italy, Spain, Serbia), which had recorded the presence of \textit{A. flavus}.  
\vspace{\baselineskip}  \\
\textbf{Heat stress}
\\
I set the value of the Heat Stress Temperature Threshold (TTHS) to 40°C same as the limiting higher temperature (DV3), and the Heat Stress Temperature Rate (THHS) to 0.0009 week\textsuperscript{-1}; I modified this value from the Haerani et al., (2020) aflatoxin in peanut's model but increased one decimal point as experts  (personal co mmunication with CLIMEX modeling expert, Darren Kriticos) suggested to take into account the heat stress in the northern parts of Africa. 
\vspace{\baselineskip} \\
\textbf{Dry stress} 
\\
I set the value of the Dry Stress Threshold (SMDS) to 0.1, the same as SM0, and set the Dry Stress Rate at -0.0002 week\textsuperscript{-1} to include the desert areas in the northern part of Africa and the middle part of Australia. 
\vspace{\baselineskip} \\
\textbf{Wet stress}
\\
For the Wet Stress Threshold (SMWS) and Wet Stress Rate (HWS), I set the value to 2°C and 0.0009 week\textsuperscript{-1}, respectively. The parameters were the same as the value from \citep{haerani2020climate} studies on aflatoxin distribution in peanuts. 


\subsubsection{Ensemble SDM}
Apart from modeling the aflatoxin production from \textit{A. flavus} in maize, I created an ensemble SDM of \textit{A. flavus} using the GBIF  occurrence data to explore the global distribution of the species. Ensemble model is a co mmon approach used in SDM that calculated outcomes from different models and created one output from all the different models \citep{hao2020testing}. The model predicted the habitat suitability of a species in a given area by giving a value from 0-1 \citep{monnier2023species}. To produce an effective SDM, I required the presence and dependable environmental data \citep{barbet2012selecting}. I used the CliMond 40 environmental variables and GBIF  occurrence data to create an ensemble SDM. I used the SSDM Package Version 0.2.8 \citep{SSDM} in RStudio to create an ensemble SDM of the species, the model consisted of seven algorithms (Table 2). I chose the algorithms based on the availability of the SSDM packages . To increase the predictive performance, I downsampled the RF algorithm and down-weighted the MARS algorithm \citep{valavi2022predictive}. Then, I set the methods to pick the pseudo-absence depending on the algorithm \citep{barbet2012selecting}, partitioning 80 percent of the data, and ran ten replications of the model. Furthermore, I set the model's weighting criteria to be the selection matrix's mean. I did not select a threshold, as defining a threshold produced the lowest amount of accuracy in the model \citep{liu2005selecting}.
\\ 
\vspace{\baselineskip}
\vspace{\baselineskip}

\begin{table}[!ht]
    \caption{Algorithms used in the ensemble SDM}
    \centering
    \begin{tabular}{lll}
        \hline
        ~ & ~ & ~  \\
        \textbf{Method} & \textbf{Description} & \textbf{R package used} \\ 
        ~ & ~ & ~  \\ \hline\hline
        ~ & ~ & ~  \\ 
        GLM & generalized linear model & stats \\ 
        GAM & generalized  addition model & mgcv \\ 
        RF & random forest (downsampled) & randomForest \\ 
        SVM & support vector machine & e1071 \\ 
        ANN & artificial neural network & nnet \\ 
        MARS & multivariate adaptive regression spline (down-weighted) & earth \\ 
        MAXENT & maximum entropy & dismo \\ 
        ~ & ~ & ~  \\ \hline
    \end{tabular}
    \label{Ensemble_Model_Algorithms}
\end{table}

\vspace{\baselineskip}


\subsection{Processing analysis}

I used RStudio Version 4.3.1 \citep{RStudio} to analyse the data in this study. After obtaining the parameters from CLIMEX, I ran the irrigated and non-irrigated output of the CLIMEX model with the 1970-2019 meteorological dataset. I extracted the NetCDF files output to RStudio \citep{RStudio} and combined the irrigated and non-irrigated scenarios to create a composite map. Then, I extracted the EI and stress variables from all the years for visualization. To identify global maize agricultural areas, I used the Map Spatial Production Allocation Model (MapSPAM) data of maize yield in 2010 \citep{international2019global} to identify areas where maizes were being grown and extracted the EI values trend from the areas. Additionally, I fitted a linear model of EI across the yearly data (1970-2019) to determine where the trends of changing EI were significant. After that, I calculated the p-values of all the linear trends and masked locations where the p values were more than 0.05 to get a confidence interval of 95 percent. Additionally, I extracted data points at locations with high maize yields and fitted a linear model on the EI values over the years.

\subsection{Future distribution}
I used two global climate models (GCMs) from the CliMond database \citep{kriticos2012climond}, which were the CSIRO-Mk3.0 (CSIRO, Australia) and MIROC-H (Centre for Climate Research, Japan) models. Then, I ran the model with the CLIMEX's paramters for aflatoxin production from \textit{A. flavus} in maize. The scenarios I used were the Special Report on Emissions Scenarios (SRES) A1B and A2 scenarios indicating specific economic growth that projects different carbon emission levels \citep{scenarios2000ipcc}. The models contained changes in every decade until 2100 \citep{kriticos2012climond}. I used four different years in this study which were 2030, 2050, 2070, and 2100.

\newpage
\section{Results}
\subsection{CLIMEX model output for aflatoxin with  occurrence data}

\subsubsection{Ecoclimatic index map}
\begin{figure*}[!ht]
	\centering
	\includegraphics[width=\textwidth]{images/composite_crop.pdf}
	\caption{EI composite map of the meteorological data from 2017 to 2019}
	\label{fig:EI composite map of the meteorological data from 2017 to 2019}
\end{figure*}
\pagebreak


I created a composite map using CLIMEX parameters with the CRU TS4 weekly data for 1970-2019 at 30’ spatial resolution \citep{mitchell2005improved}. Then, I extracted and visualized the EI of the species from 2017-2019 (Figure 4). The map indicated that most of the areas within the latitude of 30'N and -30'S were suitable for aflatoxin production. EI value has shifted throughout the years, specifically on the 40'N and -40'S latitudes, where the marginal areas shifted showing that most of Europe became marginally suitable for \textit{A. flavus}. In Australia, the marginal area shifted and provided the marginal condition for \textit{A. flavus} to grow in the central region in 2017 but not in the following years. While in South America, the eastern region of Brazil showed an optimal growth condition throughout the years. 


\subsubsection{Stress maps}
\begin{figure*}[!ht]
	\centering
	\includegraphics[width=16cm]{images/cropped_stress.pdf}
	\caption{Stress maps output for the historical climate data in CLIMEX}
	\label{fig:Stress maps output for the historical climate data in CLIMEX}
\end{figure*}

Cold stress and dry stress were the main limiting factor for the distribution of aflatoxin produced from \textit{A. flavus} in maize in the historical meteorological dataset (Figure 5). Cold stress showed that it is a significant factor in limiting \textit{A. flavus} mainly in the higher latitudes. At the same time, dry stress limited the species in the middle part of Africa and northern China. The highest level of dry stress were around the northern regions of Africa, which consisted primarily of desert biomes. Heat stress showed a medium level of impact and different areas in North Africa, while wet stress did not significantly impact the EI value of the species from this parameter.

\subsection{Linear trends of EI}

To explore the changes in suitability of \textit{A. flavus} more specifically, I extracted four coordinates with the highest amount of maize yield in mapSPAM and then correlated that location with the dataframe of extracted EI value assigned with coordinates in RStudio. All four coordinates showed a positive trend of increasing EI values. In Arizona (USA) and Xinjiang (China), the EI values increased from being marginal for \textit{A. flavus} growth to optimal over 49 years. While in Balkhash District, Kazakhstan, the increase in the level of aflatoxin was marginal throughout the dataset. The area with the second highest yield was in the North West Badiah District (Jordan) where the EI value indicated that the habitat is generally optimal for the growth of \textit{A. flavus}. While in Balkhash (Kazakhstan), the EI of the species has been in the marginal area over the years.

\begin{figure*}[!ht]
	\centering
	\includegraphics[width=16cm]{images/trend.pdf}
	\caption{Linear trend of EI from the four highest maize yield locations gathered from MapSPAM}
	\label{fig:Linear trend of EI from the four highest maize yield locations gathered from MapSPAM}
\end{figure*}

\subsection{Ensemble SDM }
\subsubsection{Ensemble SDM habitat suitability map}

\begin{figure*}[!ht]
	\centering
	\includegraphics[width=\textwidth]{images/ensemble_result.png}
	\caption{Ensemble SDM of \textit{A. flavus} habitat suitability map, the right bar represented the habitat suitability level}
	\label{fig:Ensemble SDM output}
\end{figure*}

I ran an ensemble SDM with the occurrence data of \textit{A. flavus} from the GBIF database \citep{https://doi.org/10.15468/dl.cak6s3} (Figure 7). The model showed an AUC value of 0.858, considered excellent \citep{mandrekar2010receiver}, and a kappa value of 0.685, considered substantial \citep{mchugh2012interrater}. habitat suitability trend were used to evaluate the likelihood of the species in the given area \citep{monnier2023species} was similar to CLIMEX's EI output, and the occurrence in southern Europe were within the suitable margin of the model's output. The occurrence data in the northern European regions did not correlate with high habitat suitability levels. Areas in South America showed a suitability level of approximately 0.6 in east Brazil, while the rest of the upper part of South America showed less habitat suitability level. Additionally, the model showed a higher uncertainty level (Figure 8) in the northern area of Australia and southeast of Africa. 

\begin{figure*}[!ht]
	\centering
	\includegraphics[width=\textwidth]{images/uncertainty.png}
	\caption{Ensemble SDM of \textit{A. flavus} uncertainty map}
	\label{fig:Ensemble SDM uncertainty}
\end{figure*}

\subsubsection{Model evaluation}

\begin{table}[!ht]
    \caption{Ensemble SDM of \textit{A. flavus }performance metrics}
    \centering
    \begin{tabular}{lllllll}
    \hline
        ~ & ~ & ~ & ~ & ~ & ~ & ~ \\ 
              Model & AUC  & Kappa & threshold & sensitivity & specificity & proportion correct \\ \\
        \hline \hline \\
        GLM    & 0.79 & 0.55  & 0.72      & 0.79        & 0.79        & 0.79               \\
        SVM    & 0.86 & 0.71  & 0.73      & 0.86        & 0.86        & 0.86               \\
        ANN    & 0.86 & 0.72  & 0.71      & 0.88        & 0.84        & 0.86               \\
        MARS   & 0.85 & 0.67  & 0.72      & 0.85        & 0.85        & 0.85               \\
        RF     & 0.95 & 0.89  & 0.56      & 0.95        & 0.95        & 0.95               \\
        GAM    & 0.88 & 0.74  & 0.74      & 0.88        & 0.88        & 0.88               \\
        MAXENT & 0.83 & 0.51  & 0.36      & 0.83        & 0.83        & 0.83               \\
        ~ & ~ & ~ & ~ & ~ & ~ & ~ \\
        \hline % Add spacing
    \end{tabular}
    \label{Ensemble SDM performance metrics}
\end{table}
I extracted the model's evaluation from the ensemble SDM's output (Table 3). AUC values for SVM, ANN, down-weighted MARS, and GAM AUC were considered excellent, while GLM AUC was in an acceptable range and down-sampled RF AUC value was outstanding \citep{mandrekar2010receiver}. Down-sampled RF was considered almost perfect for the kappa values, while SVM, ANN, MARS, and GAM showed substantial agreement, and GLM and MAXENT had moderate agreement \citep{mchugh2012interrater}. GLM performed the poorest amongst all the algorithms, while MAXENT's and RF's thresholds were the two lowest. The other algorithm' thresholds were generally in the same range of 0.71-0.74. Down-sampled RF algorithm showed a high sensitivity, specificity, and proportion correct amongst all the algorithms, while GLM showed the lowest amongst the analytics mentioned.

\subsection{Future distribution in CLIMEX}
\begin{figure*}[!ht]
	\centering
	\includegraphics[width=16cm]{images/MIROC_A1B.pdf}
	\caption{MIROC-H A1B future projection}
	\label{fig:MIROC-H A1B future projection}
\end{figure*}

\begin{figure*}[!ht]
	\centering
	\includegraphics[width=16cm]{images/MIROC_A2.pdf}
	\caption{MIROC-H A2 future projection}
	\label{fig:MIROC-H A2 future projection}
\end{figure*}
\pagebreak
\begin{figure*}[!ht]
	\centering
	\includegraphics[width=16cm]{images/CSIRO_A1B.pdf}
	\caption{CSIRO-Mk3.0 A1B future projection}
	\label{fig:CSIRO-Mk3.0 A1B future projection}
\end{figure*}

\begin{figure*}[!ht]
	\centering
	\includegraphics[width=16cm]{images/MIROC_A2.pdf}
	\caption{CSIRO-Mk3.0 A2 future projection}
	\label{fig:CSIRO-Mk3.0 A2 future projection}
\end{figure*}
\pagebreak
I mapped the parameters into SRES A1 and A2B scenarios using the MIROC-H and CSIRO Mk3.0 models in CliMond \citep{kriticos2012climond} for 2030, 2050, 2070, and 2100. In the SRES A2 scenarios, the marginal areas for \textit{A. flavus} extended to the northern region of Europe, indicating that regions in Sweden, Norway, and Finland were marginal for the growth of \textit{A. flavus} by 2100. For both A1B and A2 scenarios in the CSIRO Mk3.0 model, the areas where EI values were generated are generally the same from 2030 to 2070. SRES A2 scenarios in 2100 indicated that regions in eastern Russia will be marginal for the growth of \textit{A. flavus}. In Australia, there was a shift in suitability areas; where initially, in 2030, the northern region will be suitable for the production of aflatoxin from the fungus, but the model shifted, and the northwestern regions of Australia will be marginal for aflatoxin in \textit{A. flavus} by 2100 instead.

\newpage
\section{Discussion}
\subsection{Outputs of CLIMEX and ensemble SDM}
The optimal regions for aflatoxin production in maize from \textit{A. flavus} found in this study were mainly in the tropical and subtropical regions which correlated with previous studies \citep{samuel2013aflatoxin} \citep{haerani2020climate}. 
 Dry stresses in northern Africa and the cold stresses in the northern and southern latitudes limited the distribution in the dataset used.
 Particular marginal areas, such as datapoints within South Korea's agro-ecological zones, correlate with the aflatoxin level found in South Korea as Kim et al., (2013) found a low level of aflatoxins produced from maize in those areas. The marginal suitability could correlate to a low level of aflatoxin production as \textit{A. flavus} can grow there and can produce aflatoxin marginally. 
 \vspace{\baselineskip} \\
 The habitat suitability map showed a high suitability level in areas with high occurrence. Thus if the data were geographically biased, the model would likely be less effective and overfitted \citep{cosentino2021geographic}, which seemed to be the case for this dataset. The overfitting of the model can be shown in the uncertainty of the model, where there was a high uncertainty level, mainly in northern Australia; this coincides with the area where there was a high amount of  occurrence data from GBIF. The variables with the highest axes evaluation with the variation of Pearson were CliMond's bioclim variables 2 and 19 at 4.00 and 3.81 (Supplementary material 4). These variables were the mean diurnal temperature range (°C) and precipitation of the coldest quarter (mm). The two variables' high variables importance value suggested that temperature and moisture were important factors determining the  occurrence of the species. However, I used all the variables available from CliMond instead of ecologically justifying them, which likely reduced the model's accuracy \citep{kriticos2012climond}.




\subsection{Trends of EI over the years and MapSPAM maize yield}

 There were no overlapping areas with significant linear model trends (95 percent confidence) of EI with the MapSPAM maize yield data. Moreover, both maps had different spatial resolutions, so I had to modify the latitude and longitude decimal points of MapSPAM to match the EI value with the yield area. The change slightly reduced the map's accuracy as the MapSPAM database resolution was more precise than CLIMEX's raster resolution. The linear model trend of increasing EI in \\ \textit{A. flavus} in the US indicated that the temperature shift from climate change could affect corn production in the US in the future. Aflatoxin was a common occurrence in the southern region of the US compared to the midwestern area of the US, which contained the maize agricultural area known as the "Corn Belt," however, there was still the presence of these toxins in the "Corn Belt" occasionally \citep{yu2022climate}. The presence of aflatoxin was possibly from the sudden heat waves causing the temperature to be suitable for the production of the toxin occasionally \citep{an1980factors} . The location of Xinjiang province of China (43°51′ N, 87°32′ E)  contained a large area of maize cultivation \citep{wang2011changes}, the location was slightly higher in latitude where \textit{A. flavus} were usually found \citep{ehrlich2014non}, the shift in EI correlated with the future map where an increasing trend of EI indicated that the area could be optimal for \textit{A. flavus} in the future.

\subsection{Implication of different climate change scenarios on the distribution of \textit{A. flavus}}

All SRES future climate scenarios showed that northern Europe and eastern Russia regions became marginal for aflatoxin production by 2100. I compared the model's output with another study that used CLIMEX to model maize geographic suitability \citep{ramirez2017global}. MIROC-H SRES A2 CLIMEX model of maize indicated that by 2100 northern Europe and part of eastern Russia will be marginal for aflatoxin, which overlapped with the marginal areas of my output. The overlapping areas of EI could potentially be areas where \textit{A. flavus} can occur in preharvest maize in 2100, according to the scenario. The map also indicated that the US "Corn Belt" states (i.e., Indiana, Illinois, Iowa) will be more suitable for \textit{A. flavus} and the production of aflatoxin could cause a high economic impact in the regions \citep{mitchell2016potential}. All the future scenarios used also indicate that the northern regions of China will be marginal for aflatoxin likely from the decrease in dry stress levels, this overlapped with \citep{mitchell2016potential} studies where in the future scenarios, northeastern regions of China will become more suitable for the growth of maize, thus if maize is being cultivated in that area in 2100, the area could be at risk.  

\subsection{Limitations and improvements}
Both of the models I created contains various limitation and operated under the assumption that the species was not influenced by other factors apart from the environmental variables used. The occurrence data were assumed to be geographically biased; thus, the model's output indicated a high epistemic uncertainty \citep{kriticos2015climex}. CLIMEX model used in this study utilized the yearly meteorological data, which averages the temperature throughout the year; the model can not consider sudden temperature shifts (i.e., drought), which can cause environmental stress to \\ \textit{A. flavus} causing the fungus to produce the aflatoxin \citep{gallo2016effect} \citep{wu2016aflatoxin}. The study also only explored half of the aflatoxin problem as, the crops can be contaminated with aflatoxin during the postharvest stage if not kept at proper temperature \citep{kamano2022storage}. Additionally, the study did not differentiate the S and L strains of \textit{A. flavus} with the S strains producing more aflatoxin \citep{ehrlich2014non} some  occurrence of \textit{A. flavus} used in the study could be the non-aflatoxin producing strain which reduces the model's validity. Finally, apart from the temperature, various non-climatic factors could increase maize's kernel susceptibility to the colonization of aflatoxin, such as physical damage to the maize's ear by insects \citep{setamou1998effect}.

The optimal temperature values (DV1 and DV2) could be modified to 25°C and 30°C respectively, as it is within the range that most studies reported to be the optimal for aflatoxin production in other substrates and crops \citep{hassane2017influence} \citep{mannaa2017influence} \citep{norlia2020modelling} \citep{haerani2020climate}  . Modifying the values could improve the model's ability to simulate aflatoxin production and better predict the risk of aflatoxin contamination across different crops. CLIMEX's model could be further validated by setting apart a training dataset to improve the model's accuracy. The occurrence data of aflatoxin outbreaks in maize could be gathered to validate the parameters instead of using \textit{A. flavus} occurrence data in maize, which could further improve the model. The stress parameters should also be modified first, as it is used to limit the species' geographical distribution \citep{haerani2020climate}. CLIMEX \textit{A. flavus} occurrence data could be used to run an ensemble SDM or other SDM algorithms to validate the data and compared CLIMEX's AUC value \citep{early2022comparing} with other SDMs. At the same time, \textit{A. flavus} global distribution could be modeled in CLIMEX and compare the habitat suitability areas amongst the different models to compare the predictive performances. More SDM algorithms, such as BIOCLIM, could be used individually or within the ensemble SDM to determine the best method to increase the predictive performance of the pathogen system. Furthermore, due to the stochastic nature of the ensemble model, fine-tuning the algorithms used could provide higher predictive performance, and modifying the parameters in CLIMEX could increase its mechanistic performance. Regarding the trends, other regression analyses could be used to analyze and predict the future trend of EI. Furthermore, changes in suitable areas over the years could be calculated, better visualized, and compared to understand the changes more quantitatively. Further improvements on the aflatoxin parameter could be made by correlating the aflatoxin production level of both temperature and moisture, as aflatoxin can be produced optimally at different temperature ranges given specific water activity \citep{mannaa2017influence}. Finally, further understanding of the species and ecologically justified the environmental variables from CliMond \citep{kriticos2012climond} to only variables related to the distribution of \textit{A. flavus} could improve the predictive efficacy of the model.

\newpage
\section{Conclusion}
I modeled the distribution of aflatoxin production in \textit{A. flavus} in maize to determine the trends of changes in suitability within the agricultural areas using historical and future datasets. There was a significant trend of change in the habitat suitability shown in the historical dataset; by 2100, under SRES A2 climate scenario, areas in northern Europe and eastern Russia would be suitable for \\ \textit{A. flavus} to inhabit, given the parameters used. Comparing this future scenario map with the CLIMEX model of maize suitability areas by Ramirez-Cabral et al., (2017) showed that maize cultivation could be at risk of aflatoxin produced by \texit{A. flavus} if moved in the areas mentioned. Aflatoxin is a potent mycotoxin responsible for many liver cancer cases each year \citep{liu2010global}. A complete understanding of different aspects of the species, be it ecologically or genetically, could be useful in helping us understand and reduce the burden of its impact on health. There is also a high economic benefit from reducing aflatoxin levels as this could increase food security globally \citep{gbashi2018socio}. Not many pathogen models were being parameterized on CLIMEX, and this study models the species \textit{A. flavus} within the context of maize and able to explore its distribution. Despite the limitations, various highly significant variables were considered, making this a reasonably reliable model to build upon in future studies. The distribution of this fungus was highly dependent on climate; thus, by understanding its distribution fully, we could mitigate its effect, reduce the global burden of crop loss, and reduce health impacts from the toxin produced by the species.

\pagebreak
\section{References}
\bibliography{irrbibfile.bib}
\pagebreak

\pagenumbering{roman}

\section{Supplementary Information}
\subsection{EI historical data map}
\begin{figure*}[!ht]
	\centering
	\includegraphics[width=16cm]{images/EI_map.pdf}
	\caption{Historical EI map for irrigated scenario}
	\label{fig:Historical EI map for irrigated scenario}
\end{figure*}


\pagebreak
\subsection{CLIMEX parameter sensitivity}
\begin{table}[ht]
\caption{CLIMEX Parameters sensitivity for each parameter}
\begin{tabular}{lllllllllllll}
\hline
Param. & Low   & Default & High  & Run   & Range & EI   & GI   & MI    & TI   & DS   & HS   & CS   \\ \hline
\\
SM0   & 0.00  & 0.10    & 0.20  & 1.00  & 3.93  & 4.57 & 5.18 & 23.18 & 0.00 & 0.00 & 0.00 & 0.00 \\
SM1   & 0.10  & 0.20    & 0.30  & 2.00  & 0.80  & 3.61 & 4.17 & 20.87 & 0.00 & 0.00 & 0.00 & 0.00 \\
SM2   & 0.70  & 0.80    & 0.90  & 3.00  & 0.01  & 1.61 & 1.65 & 4.88  & 0.00 & 0.00 & 0.00 & 0.00 \\
SM3   & 1.90  & 2.00    & 2.10  & 4.00  & 0.01  & 1.83 & 1.84 & 2.90  & 0.00 & 0.00 & 0.00 & 0.00 \\
DV0   & 9.00  & 10.00   & 11.00 & 5.00  & 0.16  & 0.83 & 1.33 & 0.00  & 2.04 & 0.00 & 0.00 & 0.00 \\
DV1   & 29.00 & 30.00   & 31.00 & 6.00  & 0.05  & 1.63 & 1.74 & 0.00  & 3.44 & 0.00 & 0.00 & 0.00 \\
DV2   & 34.00 & 35.00   & 36.00 & 7.00  & 0.04  & 0.90 & 0.91 & 0.00  & 2.52 & 0.00 & 0.00 & 0.00 \\
DV3   & 39.00 & 40.00   & 41.00 & 8.00  & 0.06  & 0.45 & 0.45 & 0.00  & 2.13 & 0.00 & 0.00 & 0.00 \\
TTCS  & 9.00  & 10.00   & 11.00 & 9.00  & 1.21  & 1.13 & 0.00 & 0.00  & 0.00 & 0.00 & 0.00 & 7.23 \\
THCS  & 0.00  & 0.00    & 0.00  & 10.00 & 1.30  & 0.99 & 0.00 & 0.00  & 0.00 & 0.00 & 0.00 & 6.43 \\
TTHS  & 39.00 & 40.00   & 41.00 & 11.00 & 0.00  & 0.04 & 0.00 & 0.00  & 0.00 & 0.00 & 0.23 & 0.00 \\
THHS  & 0.00  & 0.00    & 0.00  & 12.00 & 0.00  & 0.02 & 0.00 & 0.00  & 0.00 & 0.00 & 0.10 & 0.00 \\
SMDS  & 0.00  & 0.10    & 0.20  & 13.00 & 0.07  & 0.21 & 0.00 & 0.00  & 0.00 & 8.99 & 0.00 & 0.00 \\
HDS   & 0.00  & 0.00    & 0.00  & 14.00 & 0.00  & 0.06 & 0.00 & 0.00  & 0.00 & 1.32 & 0.00 & 0.00 \\
SMWS  & 1.90  & 2.00    & 2.10  & 15.00 & 0.00  & 0.09 & 0.00 & 0.00  & 0.00 & 0.00 & 0.00 & 0.00 \\
HWS   & 0.00  & 0.00    & 0.00  & 16.00 & 0.00  & 0.05 & 0.00 & 0.00  & 0.00 & 0.00 & 0.00 & 0.00
\end{tabular}
\hline
\vspace{1cm}
\pagebreak
\label{tab:CLIMEX Parameters Sensitivity}
\end{table}

\subsection{CliMond bioclim variables description}
\begin{table}
\caption{Description of CliMond variables}\citep{hutchinson2009anuclim} \citep{kriticos2012climond} \citep{kriticos2014extending}
\centering
\begin{tabular}{ht}
\\
Variable Number & Variable \\
\hline
Bio01 & Annual mean temperature (°C) \\
Bio02 & Mean diurnal temperature range (°C) \\
Bio03 & Isothermality (Bio02 ÷ Bio07) \\
Bio04 & Temperature seasonality (C of V) \\
Bio05 & Max temperature of warmest week (°C) \\
Bio06 & Min temperature of coldest week (°C) \\
Bio07 & Temperature annual range (Bio05-Bio06) (°C) \\
Bio08 & Mean temperature of wettest quarter (°C) \\
Bio09 & Mean temperature of driest quarter (°C) \\
Bio10 & Mean temperature of warmest quarter (°C) \\
Bio11 & Mean temperature of coldest quarter (°C) \\
Bio12 & Annual precipitation (mm) \\
Bio13 & Precipitation of wettest week (mm) \\
Bio14 & Precipitation of driest week (mm) \\
Bio15 & Precipitation seasonality (C of V) \\
Bio16 & Precipitation of wettest quarter (mm) \\
Bio17 & Precipitation of driest quarter (mm) \\
Bio18 & Precipitation of warmest quarter (mm) \\
Bio19 & Precipitation of coldest quarter (mm) \\
Bio20 & Annual mean radiation (W  m$^2$) \\
Bio21 & Highest weekly radiation (W  m$^2$) \\
Bio22 & Lowest weekly radiation (W  m$^2$) \\
Bio23 & Radiation seasonality (C of V) \\
Bio24 & Radiation of wettest quarter (W  m$^2$) \\
Bio25 & Radiation of driest quarter (W  m$^2$) \\
Bio26 & Radiation of warmest quarter (W  m$^2$) \\
Bio27 & Radiation of coldest quarter (W  m$^2$) \\
Bio28 & Annual mean moisture index \\
Bio29 & Highest weekly moisture index \\
Bio30 & Lowest weekly moisture index \\
Bio31 & Moisture index seasonality (C of V) \\
Bio32 & Mean moisture index of wettest quarter \\
Bio33 & Mean moisture index of driest quarter \\
Bio34 & Mean moisture index of warmest quarter \\
Bio35 & Mean moisture index of coldest quarter \\
Bio36 & First principal component of the first 35 Bioclim variables \\
Bio37 & Second principal component of the first 35 Bioclim variables \\
Bio38 & Third principal component of the first 35 Bioclim variables \\
Bio39 & Fourth principal component of the first 35 Bioclim variables \\
Bio40 & Fifth principal component of the first 35 Bioclim variables \\
\hline
\end{tabular}
\label{tab:variables}
\end{table}
\pagebreak

\subsection{Ensemble SDM variables importance}
\begin{figure*}[!ht]
	\centering
	\includegraphics[width=\textwidth]{images/var_importance.png}
	\caption{Ensemble SDM variables importance}
	\label{fig:Ensemble SDM variables importance}
\end{figure*}

\subsection{Ensemble SDM algorithm heatmap}
\begin{figure*}[!ht]
	\centering
	\includegraphics[width=10cm]{images/ensemble_corr.png}
	\caption{Ensemble SDM correlation heatmap}
	\label{fig:Ensemble SDM correlation heatmap}
\end{figure*}

\pagebreak

\subsection{Ensemble model algorithms evaluation}
\begin{figure*}[!ht]
	\centering
	\includegraphics[width=10cm]{images/mod_eva.png}
	\caption{Ensemble model algorithms evaluation, the color indicates the heatmap intensity}
	\label{fig:Ensemble model algorithms evaluation}
\end{figure*}

\subsection{Trend significance of the composite map in CLIMEX}
\begin{figure*}[!ht]
	\centering
	\includegraphics[width=10cm]{images/trend_sig.png}
	\caption{Trend significance of the composite map in CLIMEX, the right legend explains the habitat suitability level}
	\label{fig:Trend significance of the composite map in CLIMEX}
\end{figure*}

\pagebreak
\subsection{Output of CLIMEX suitabiliy of maize from \citep{ramirez2017global}}
\begin{figure*}[!ht]
	\centering
	\includegraphics[width=\textwidth]{images/ramirez's.png}
	\caption{CLIMEX maize suitability output from \citep{ramirez2017global}}
	\label{fig:CLIMEX maize output}
\end{figure*}




\end{document}          