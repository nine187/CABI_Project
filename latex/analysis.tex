\subsection{Outputs of CLIMEX and ensemble SDM}
The optimal regions for aflatoxin production in maize from \textit{A. flavus} found in this study were mainly in the tropical and subtropical regions which correlated with previous studies \citep{samuel2013aflatoxin} \citep{haerani2020climate}. 
 Dry stresses in northern Africa and the cold stresses in the northern and southern latitudes limited the distribution in the dataset used.
 Particular marginal areas, such as datapoints within South Korea's agro-ecological zones, correlate with the aflatoxin level found in South Korea as Kim et al., (2013) found a low level of aflatoxins produced from maize in those areas. The marginal suitability could correlate to a low level of aflatoxin production as \textit{A. flavus} can grow there and can produce aflatoxin marginally. 
 \vspace{\baselineskip} \\
 The habitat suitability map showed a high suitability level in areas with high occurrence. Thus if the data were geographically biased, the model would likely be less effective and overfitted \citep{cosentino2021geographic}, which seemed to be the case for this dataset. The overfitting of the model can be shown in the uncertainty of the model, where there was a high uncertainty level, mainly in northern Australia; this coincides with the area where there was a high amount of  occurrence data from GBIF. The variables with the highest axes evaluation with the variation of Pearson were CliMond's bioclim variables 2 and 19 at 4.00 and 3.81 (Supplementary material 4). These variables were the mean diurnal temperature range (°C) and precipitation of the coldest quarter (mm). The two variables' high variables importance value suggested that temperature and moisture were important factors determining the  occurrence of the species. However, I used all the variables available from CliMond instead of ecologically justifying them, which likely reduced the model's accuracy \citep{kriticos2012climond}.




\subsection{Trends of EI over the years and MapSPAM maize yield}

 There were no overlapping areas with significant linear model trends (95 percent confidence) of EI with the MapSPAM maize yield data. Moreover, both maps had different spatial resolutions, so I had to modify the latitude and longitude decimal points of MapSPAM to match the EI value with the yield area. The change slightly reduced the map's accuracy as the MapSPAM database resolution was more precise than CLIMEX's raster resolution. The linear model trend of increasing EI in \\ \textit{A. flavus} in the US indicated that the temperature shift from climate change could affect corn production in the US in the future. Aflatoxin was a common occurrence in the southern region of the US compared to the midwestern area of the US, which contained the maize agricultural area known as the "Corn Belt," however, there was still the presence of these toxins in the "Corn Belt" occasionally \citep{yu2022climate}. The presence of aflatoxin was possibly from the sudden heat waves causing the temperature to be suitable for the production of the toxin occasionally \citep{an1980factors} . The location of Xinjiang province of China (43°51′ N, 87°32′ E)  contained a large area of maize cultivation \citep{wang2011changes}, the location was slightly higher in latitude where \textit{A. flavus} were usually found \citep{ehrlich2014non}, the shift in EI correlated with the future map where an increasing trend of EI indicated that the area could be optimal for \textit{A. flavus} in the future.

\subsection{Implication of different climate change scenarios on the distribution of \textit{A. flavus}}

All SRES future climate scenarios showed that northern Europe and eastern Russia regions became marginal for aflatoxin production by 2100. I compared the model's output with another study that used CLIMEX to model maize geographic suitability \citep{ramirez2017global}. MIROC-H SRES A2 CLIMEX model of maize indicated that by 2100 northern Europe and part of eastern Russia will be marginal for aflatoxin, which overlapped with the marginal areas of my output. The overlapping areas of EI could potentially be areas where \textit{A. flavus} can occur in preharvest maize in 2100, according to the scenario. The map also indicated that the US "Corn Belt" states (i.e., Indiana, Illinois, Iowa) will be more suitable for \textit{A. flavus} and the production of aflatoxin could cause a high economic impact in the regions \citep{mitchell2016potential}. All the future scenarios used also indicate that the northern regions of China will be marginal for aflatoxin likely from the decrease in dry stress levels, this overlapped with \citep{mitchell2016potential} studies where in the future scenarios, northeastern regions of China will become more suitable for the growth of maize, thus if maize is being cultivated in that area in 2100, the area could be at risk.  

\subsection{Limitations and improvements}
Both of the models I created contains various limitation and operated under the assumption that the species was not influenced by other factors apart from the environmental variables used. The occurrence data were assumed to be geographically biased; thus, the model's output indicated a high epistemic uncertainty \citep{kriticos2015climex}. CLIMEX model used in this study utilized the yearly meteorological data, which averages the temperature throughout the year; the model can not consider sudden temperature shifts (i.e., drought), which can cause environmental stress to \\ \textit{A. flavus} causing the fungus to produce the aflatoxin \citep{gallo2016effect} \citep{wu2016aflatoxin}. The study also only explored half of the aflatoxin problem as, the crops can be contaminated with aflatoxin during the postharvest stage if not kept at proper temperature \citep{kamano2022storage}. Additionally, the study did not differentiate the S and L strains of \textit{A. flavus} with the S strains producing more aflatoxin \citep{ehrlich2014non} some  occurrence of \textit{A. flavus} used in the study could be the non-aflatoxin producing strain which reduces the model's validity. Finally, apart from the temperature, various non-climatic factors could increase maize's kernel susceptibility to the colonization of aflatoxin, such as physical damage to the maize's ear by insects \citep{setamou1998effect}.

The optimal temperature values (DV1 and DV2) could be modified to 25°C and 30°C respectively, as it is within the range that most studies reported to be the optimal for aflatoxin production in other substrates and crops \citep{hassane2017influence} \citep{mannaa2017influence} \citep{norlia2020modelling} \citep{haerani2020climate}  . Modifying the values could improve the model's ability to simulate aflatoxin production and better predict the risk of aflatoxin contamination across different crops. CLIMEX's model could be further validated by setting apart a training dataset to improve the model's accuracy. The occurrence data of aflatoxin outbreaks in maize could be gathered to validate the parameters instead of using \textit{A. flavus} occurrence data in maize, which could further improve the model. The stress parameters should also be modified first, as it is used to limit the species' geographical distribution \citep{haerani2020climate}. CLIMEX \textit{A. flavus} occurrence data could be used to run an ensemble SDM or other SDM algorithms to validate the data and compared CLIMEX's AUC value \citep{early2022comparing} with other SDMs. At the same time, \textit{A. flavus} global distribution could be modeled in CLIMEX and compare the habitat suitability areas amongst the different models to compare the predictive performances. More SDM algorithms, such as BIOCLIM, could be used individually or within the ensemble SDM to determine the best method to increase the predictive performance of the pathogen system. Furthermore, due to the stochastic nature of the ensemble model, fine-tuning the algorithms used could provide higher predictive performance, and modifying the parameters in CLIMEX could increase its mechanistic performance. Regarding the trends, other regression analyses could be used to analyze and predict the future trend of EI. Furthermore, changes in suitable areas over the years could be calculated, better visualized, and compared to understand the changes more quantitatively. Further improvements on the aflatoxin parameter could be made by correlating the aflatoxin production level of both temperature and moisture, as aflatoxin can be produced optimally at different temperature ranges given specific water activity \citep{mannaa2017influence}. Finally, further understanding of the species and ecologically justified the environmental variables from CliMond \citep{kriticos2012climond} to only variables related to the distribution of \textit{A. flavus} could improve the predictive efficacy of the model.