Aflatoxins is a group of secondary metabolites mycotoxins commonly found in tropical and subtropical regions \citepp{wu2015global}. Amongst all the aflatoxins, aflatoxin $B_{1}$ ($AFB_{1}$) is considered the most dangerous due to its hepatocarcinogenic property, \citep{gizachew2019aflatoxin} with the International Agency for Research on Cancer (IARC) classifying it as a group 1 carcinogen \citep{international2012review}. These mycotoxins are mainly produced by two species of fungus: \textit{Aspergillus flavus} producing aflatoxin $B_{1}$ and $B_{2}$ and \textit{Aspergillus parasiticus} producing aflatoxin $G_{1}$, $G_{2}$, and $M_{1}$ \citep{10.1079/cabicompendium.7432}. Aflatoxin was first discovered in the United Kingdom in 1961 after investigating mass turkey death caused by aflatoxin-contaminated groundnut feed \citep{aflatoxins1979other}. The mycotoxin is an agricultural and economic \citep{wu2015global} concern as it can contaminate a wide range of host plants, including various staple crop species \citep{10.1079/cabicompendium.7432}. For this reason, many studies \citep{chauhan2015improved} \citep{bernaldez2017influence}  had been conducted to investigate this toxin.

\begin{figure*}[!ht]
    \centering
    \begin{minipage}{10cm}
        \centering
        \includegraphics[width=\linewidth]{images/infected_corn.jpg}
        \caption{\textit{A. flavus} infected maize \\ Image Source: Denis C. McGee/Iowa State University \citep{10.1079/cabicompendium.7432} [last accessed 18/08/2023]}
        \label{fig:aflavus-infected-maize}
    \end{minipage}
\end{figure*}

One of the main producers of aflatoxin, the fungus species: \textit{A. flavus} is an airbone saprophyte fungus \citep{goldblatt2012aflatoxin} widely spread throughout the world in various forms \citep{diener1987epidemiology}. \textit{A. flavus} consists of 2 main morphological strains group: the S strain and the L strain, distinguishable by their sclerotia size \citep{amaike2011aspergillus}; due to the morpohological difference, both strains have slightly different ecological niches and aflatoxin production levels, resulting in the different toxigenicity level geographically \citep{10.1079/cabicompendium.7432}. The fungus can infect various crops during the pre or postharvest stage and produce a bright greenish-yellow fluorescent (BGYF) color (Figure 1), indicating the presence of the fungus; however, this does not prove that there is the presence of aflatoxin \citep{yao2006hyperspectral}. \vspace{\baselineskip} 

In countries that exports high amount of staple food crops, maize and peanuts are the two of the most contaminated from the aflatoxin \citep{scussel2006study}. The contaminations are usually dominated by two species of fungus \textit{A. parasiticus} and \textit{A. flavus} amongst different species of crops \citep{kumar2017aflatoxins}.  Thus for economic and health concerns, there has been much focus on the aflatoxin and the optimal conditions which could stimulate the growth of the fungus in pre and postharvest agricultural host plants \citep{chauhan2015improved} \citep{gizachew2019aflatoxin}  \citep{liu2021physico}. Moreover, due to the nature of the aflatoxin production being highly related to temperature \citep{arrus2005aflatoxin}, there has been extensive efforts \citep{haerani2020climate} \citep{valencia2020environment} in exploring a shift of the production range of these secondary carcinogenic metabolites globally due to climate change \citep{haerani2020climate}. The global temperature rise will change the species ecological niche, and the increased level of carbon dioxide from climate change has been associated with an increased level of aflatoxin production in maize \citep{gilbert2017carbon}. 
\vspace{\baselineskip} \\
With the shifting range of aflatoxin in the future, a better understanding of one of the main aflatoxin-producing species' \textit{A. flavus} distributions and their distribution in various crops are required. In order to do so, the use of different approaches of modeling to identify and verify areas where the toxin threatens crops is crucial. Therefore, I attempted to model the global range of the pathogen aflatoxin produced from \textit{A. flavus} in one of the major crops that are affected by aflatoxin contamination: maize and the global range of the fungus \textit{A. flavus }. In order to predict the global range of the aflatoxin in maize from the fungus, I used the mechanistic modeling tool: CLIMEX to predict the distribution of the species as the model uses the ecological niche of a species to predict the distribution. For the global distribution of \textit{A. flavus}, I used an ensemble Species Distribution modeling (SDM) technique comprising different algorithms with varying reliability \citep{valavi2022predictive} to determine the global range of \textit{A. flavus}. 
\vspace{\baselineskip} \\ 
Alongside the models, I used the occurrence and environmental data available in the literature and online databases (i.e., GBIF, CliMond) to validify and predict the distribution. I also explored the trends of the changes in the ecological niche of the fungus in high maize yield areas over the years and explored potential areas that involved significant changes in suitability. Apart from modeling the distribution with historical meteorological data, I applied the parameters to future climate models to explore areas where the potential change in the trend of the ecological niche of \textit{A. flavus} could occur.